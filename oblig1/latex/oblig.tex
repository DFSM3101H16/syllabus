\documentclass[letterpaper,12pt]{article}
\usepackage{tabularx,amsmath,boxedminipage,graphicx}
\usepackage[margin=1in,letterpaper]{geometry} % this shaves off default margins which are too big
\usepackage{cite}
\usepackage[squaren, derived]{SIunits} 
\usepackage{rotating} % too rotate figures

\usepackage{listings}
\lstset{
  language=matlab,
  basicstyle=\ttfamily
}

\usepackage[utf8]{inputenc}
\usepackage[final]{hyperref} % adds hyper links inside the generated pdf file
\hypersetup{
  colorlinks=true,       % false: boxed links; true: colored links
  linkcolor=blue,          % color of internal links
  citecolor=blue,        % color of links to bibliography
  filecolor=magenta,      % color of file links
  urlcolor=blue         
}

\begin{document}

\title{Obligmal}
\author{Eirik Ovrum}
\date{\today}
\maketitle

\section{Introduksjon}
``The avengers of Ultron: The sliding game'' er et spill hvor man skal prøve å skli en boks opp et skråplan så langt som mulig uten å falle over enden.

\section{Fysikken}
Fysikken som ligger til grunn for spillet er friksjon og tyngdekraft på en kloss på et skråplan. 

\subsection{Krefter på klossen}
I figur \ref{fig:kloss} ser vi systemet satt opp, med kreftene som virker på klossen. Her peker friksjonskrafta feil vei, siden klossen vår skal skli oppover på skråplanet. 



\begin{figure}[ht!] 
  \label{fig:kloss}
  \begin{center}
    \begin{turn}{0} % rotering
        \includegraphics[scale=0.2]{boks} % scale er størrelse ift orig
    \end{turn}
  \end{center}
  \caption{Kloss på skråplan. Her er er $f$ friksjonskrafta når klossen sklir nedover skråplanet. I spillet derimot så sklir klossen oppover skråplanet, og $\vec f$ peker da motsatt vei, nedover skråplanet.}
\end{figure}

Kreftene som virker på klossen er tyngdekrafta ($\vec G$), normalkrafta fra underlaget ($\vec N$) og friksjonskrafta fra underlaget ($\vec f$).

\subsection{Variabler og input til spillet}
Det tingene vi må vite om klossen før vi kan simulere bevegelsen dens er 

\begin{enumerate}
\item Massen, $m$, $[m]=\kilogram$
\item Vinkelen skråplanet danner med horisontalen, $\theta$
\item Startfarten oppover skråplanet, $v_0$, $[v_0]=\meter\per\second$
\item Den dynamiske friksjonskoeffisienten mellom klossen og underlaget, $\mu_k$
\item Lengden av skråplanet, $L$, $[L]=\meter$.
\end{enumerate}



\subsection{Løsningsmetode}
Måten vi simulerer bevegelsen på, er å bruke Newtons andre lov som gir akselerasjonen ut i fra kreftene. Og fra akselerasjonen, samt startfart og startposisjon, finner vi farta og posisjonen for ethvert tidspunkt. 

Først velger vi oss et kartesisk koordinatsystem, med positiv $x$-retning oppover skråplanet, og med positiv $y$-retning normalt oppover fra skråplanet.
Så dekomponerere vi kreftene i de to retningene og setter opp Newtons andre lov for hver av retningene.

Først for $y$-retning, 
\begin{equation}
\Sigma F_y = N-G_y = N - G\cos\theta= 0 \Rightarrow N=G\cos\theta
\end{equation}
som gir oss størrelsen på normalkrafta, siden akselerasjonen i $y$-retning er null. Som igjen gir oss friksjonskrafta,

\begin{equation}
f = \mu_k N = \mu_k mg\cos\theta.
\end{equation}

Newtons andre lov i $x$-retning gir oss så akselerasjonen i $x$-retning. 

\[
\Sigma F_x = -f -G\sin\theta = - mg( \mu_k \cos\theta +\sin\theta) = m a_x 
\]
\begin{equation}
\Rightarrow a_x = - g( \mu_k \cos\theta + \sin\theta).
\end{equation}

Løsningen av dette problemet kan finnes i enhver fysikkbok for ingeniørstudenter eller i boka vi bruker i dette kurset, \cite{book}.

\subsection{Bevegelsesligningene}
Vi har her bevegelse med konstant akselerasjon, og trenger dermed ikke å bruke en numerisk løser for å integrere opp posisjonen og farten fra akselerasjonen. Vi har et tilfelle med kjent startposisjon: $x_0=\unit{0}{\meter}$, kjent startfart $v_0$ i $x$-retning. Samtidig så kjenner vi akselerasjonen, og vi finner da posisjonen som en funksjon av tiden:

\begin{equation}
x = x_0 + v_0 t + \frac{1}{2}a_x t^2.
\end{equation}

Trenger vi farta, så er den 
\begin{equation}
v=v_0 +a_x t.
\end{equation}


\subsection{Løsning for gitte startverdier}
For å vise at spillet gir riktig fysikk så har jeg også løst problemet i octave, og tar med en figur som viser den riktige oppførselen til klossen.
Under følger et utdrag fra octave fila, hele fila ligger i repositoriet under navnet \emph{sb1.m}.

\begin{lstlisting}

% I x' retning bruker vi N2.
a_x2 = 1/m*( G2(1) - f ); % akselerasjonen i x' retning, m/s^2

t = linspace(t_0, dt*t_N, t_N);
x2= zeros(t_N, 1)'; % En kolonnevektor med t_N nuller, x'
                    % posisjonen.
x= zeros(t_N, 1)'; % x posisjonen
y= zeros(t_N, 1)'; % y posisjonen


x2(1) = 0;
for i=2:t_N
    x2(i) = v_0 * t(i) + 1/2 * a_x2 * t(i)^2;
end

plot(t,x2)
\end{lstlisting}


Den figuren som kalles ved koden over er denne
\begin{figure}[ht!] 
  \label{fig:kloss}
  \begin{center}
    \begin{turn}{0} % rotering
        \includegraphics[scale=0.4]{octave_plot1} % scale er størrelse ift orig
    \end{turn}
  \end{center}
  \caption{Her vises posisjonen i $x$-retning for klossen som en funksjon av tiden. Den vertikale aksen i figuren er altså hvor langt oppover skråplanet klossen har kommet. Som vi ser, så stopper den på rundt $\unit{8}{\meter}$, mens skråplanet bare er $\unit{1}{\meter}$ langt, og vi taper dermed spillet med de startverdiene vi har brukt her. }
\end{figure}



\section{Implementasjon}
Hvordan har fysikken blitt implementert? Gjerne med pseudokode eller faktisk kode. Forklar hvordan dere har fått de ligningene som gir korrekt løsning inn i unity.


\bibliography{oblig}
\bibliographystyle{plain}

\end{document}